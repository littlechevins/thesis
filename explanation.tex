\chapter{Explanation??}\label{ch:explanation}

\section{How to solve??}

To tackle the problem of blockchain overhead, we propose an Anonymous Routing Backbone (ARB). In conventional energy trading approaches, transactions are broadcasted using the target blockchain while incurs packet overheads and delays between negotiating nodes. We aim reduce this bottleneck with a solution of participating nodes in the blockchain that have significant resources. We let these nodes be substations or controller centers and they serve as the backbone nodes which are responsible for the routing of negotiation packets between communication nodes. 

\section{Initialisation}

\subsection{Establishing Neighbours with Handshakes}

Each node begins by sending a IRQN packet to all of it's potential neighbours. Let us refer to the original broadcasting node as 'Node \textit{k}' and the potential neighbour node as 'Node \textit{j}' -we will focus on one neighbouring node- Node \textit{j} will then perform a check on the validity of the incoming packet. If the packet is validated, Node \textit{j} will declare itself as '2-way' with respect to Node \textit{k}. It will then reply with a IRQNR packet to Node \textit{k} which in turn will perform a similar validity check. Node \textit{k} will then set itself as '2-way' and broadcast a final confirmation packet -CRQNR- to Node \textit{j} which enforces a neighbouring state between the two nodes.

These Handshakes happen simultaneously for all nodes in a potential network. If a node has already sent a IRQN packet to a specific node \textit{x}, it will discard all IRQN packets from node \textit{x} since it is waiting for a IRQNR reply. This lessens the computational load on nodes during the initial setup phase. All handshake packets are ignored between two nodes if they have already established a neighbouring state between each other.

\subsection{Building the backbone map}

Once a node has successfully established a neighbouring state between all its neighbours it will begin to broadcast a 'LSA' packet to all its neighbours. This packet contains information on the current nodes knowledge of the network. Outlined in \textbf{**FIGURE**}, we can see that when another node receives this 'LSA' packet, it will rebuild its knowledge of the network and then pass this updated information onto its neighbouring nodes in the next pass of the 'LSA'. After the Handshake stage, all nodes are in a perpetual state of broadcasting. This is because no node can truely realise the full size of a given network since it will not know which node is the last in a map. An always broadcasting system allows for new bNodes to be quickly added into the network map and for new information to be rapidly propagated throughout the network.


\subsection{Building DHT}

Occurring simultaneously with 4.2.2, the DHT is rebuilt every time a the network map is updated for a given node. The reasoning for this scaled down solution is outlined in Chapter 5: Improvements. 

In small scale systems, the key of a given node will be quite small as outlined in \textbf{**FFIGURE**}, we use a Routing Byte (RB) of 1. Routing bytes are the most significant figures of a given public key. As the routing bytes increase (due to system load), we must also increase the RB to try minimise the load on each node. 

In order to do this, a base 36 system was developed for calculations with an alphanumerical system.

\begin{center}
    0123456789abcedfghijklmnopqrstuvwxyz
\end{center}

The RB is used to calculate the exact divide between the amount of nodes and the length of the RB

\begin{center}
    $$\frac{36^{RB}}{\text{n}} - 1$
\end{center}

Where n is the number of nodes in the network. We then encoded this number and the next one in sequence into base36 and then use it to split each node into a series of higher and lower key pairs as shown in \textbf{**FIGURE**}


As each new backbone comes in, we rebuild the dht table


\section{During the run??}

\subsection{New End Node}

\subsection{New Backbone Node}

\section{Features}

\subsection{Denial of Service}ss

\section{Implementation}

The nodes are built on the Python Flask framework for reliability and speed. It uses the Requests library for sending and receiving packets.


