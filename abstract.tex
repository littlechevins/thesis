\chapter*{Abstract}\label{abstract}
In recent years, blockchain has attracted tremendous attention from both academia and industry to provide a distributed, secure, trusted, and private framework for energy trading in smart grids. However, existing blockchain solutions suffer from lack of pri- vacy, processing and packet overheads, and reliance on Trusted Third Parties (TTP). To address these challenges, researchers at CSE have proposed a Secure Private Blockchain- based (SPB) framework. SPB enables the energy producers and consumers to directly negotiate the energy price. To reduce the associated packet overhead, a routing method is employed which routes packets based on the destination Public Key (PK). SPB elim- inates the need for TTP by introducing atomic meta-transactions. These transactions are visible to the blockchain participants only after all of them are generated. Thus, if any of the participants does not commit to its tasks in a pre-defined time, then the energy trade expires and all generated transactions are treated as invalid. The smart meter of the consumer confirms receipt of energy by generating an Energy Receipt Confirmation (ERC). To verify that the ERC is generated by a genuine smart meter, SPB proposes a private authentication mechanism which protects the privacy of the meter owner by breaking the links between transactions generated by the meter. This thesis will focus on developing a full-stack implementation of SPB on the Ethereum blockchain. The implementation will be evaluated over a real testbed comprised of solar panels, batteries and smart meters. A thorough performance evaluation and security analysis will be performed.